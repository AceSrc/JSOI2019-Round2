%%%%%%%%%%%%%%%%%%%%%%%%%%%%%%%%%%%%%%%%%
% Beamer Presentation
% LaTeX Template
% Version 1.0 (10/11/12)
%
% This template has been downloaded from:
% http://www.LaTeXTemplates.com
%
% License:
% CC BY-NC-SA 3.0 (http://creativecommons.org/licenses/by-nc-sa/3.0/)
%
%%%%%%%%%%%%%%%%%%%%%%%%%%%%%%%%%%%%%%%%%

%----------------------------------------------------------------------------------------
%	PACKAGES AND THEMES
%----------------------------------------------------------------------------------------

\documentclass{beamer}

\mode<presentation> {

% The Beamer class comes with a number of default slide themes
% which change the colors and layouts of slides. Below this is a list
% of all the themes, uncomment each in turn to see what they look like.

%\usetheme{default}
%\usetheme{AnnArbor}
%\usetheme{Antibes}
%\usetheme{Bergen}
%\usetheme{Berkeley}
%\usetheme{Berlin}
%\usetheme{Boadilla}
%\usetheme{CambridgeUS}
%\usetheme{Copenhagen}
%\usetheme{Darmstadt}
%\usetheme{Dresden}
%\usetheme{Frankfurt}
%\usetheme{Goettingen}
%\usetheme{Hannover}
%\usetheme{Ilmenau}
%\usetheme{JuanLesPins}
%\usetheme{Luebeck}
\usetheme{Madrid}
%\usetheme{Malmoe}
%\usetheme{Marburg}
%\usetheme{Montpellier}
%\usetheme{PaloAlto}
%\usetheme{Pittsburgh}
%\usetheme{Rochester}
%\usetheme{Singapore}
%\usetheme{Szeged}
%\usetheme{Warsaw}

% As well as themes, the Beamer class has a number of color themes
% for any slide theme. Uncomment each of these in turn to see how it
% changes the colors of your current slide theme.

%\usecolortheme{albatross}
%\usecolortheme{beaver}
%\usecolortheme{beetle}
%\usecolortheme{crane}
%\usecolortheme{dolphin}
%\usecolortheme{dove}
%\usecolortheme{fly}
%\usecolortheme{lily}
%\usecolortheme{orchid}
%\usecolortheme{rose}
%\usecolortheme{seagull}
%\usecolortheme{seahorse}
%\usecolortheme{whale}
%\usecolortheme{wolverine}

%\setbeamertemplate{footline} % To remove the footer line in all slides uncomment this line
%\setbeamertemplate{footline}[page number] % To replace the footer line in all slides with a simple slide count uncomment this line

%\setbeamertemplate{navigation symbols}{} % To remove the navigation symbols from the bottom of all slides uncomment this line
}

\usepackage[english]{babel}
\usepackage[utf8]{inputenc}
\usepackage{amsmath}
\usepackage{fontspec, xcolor}

\usepackage[slantfont,boldfont]{xeCJK}

\usepackage{graphicx} % Allows including images
\usepackage{booktabs} % Allows the use of \toprule, \midrule and \bottomrule in tables

%----------------------------------------------------------------------------------------
%	TITLE PAGE
%----------------------------------------------------------------------------------------

\title[Short title]{JSOI2019 Round2 评讲} % The short title appears at the bottom of every slide, the full title is only on the title page

\author{AceSrc} % Your name
\institute[] % Your institution as it will appear on the bottom of every slide, may be shorthand to save space
{
% Nanjing University \\ % Your institution for the title page
\medskip
\textit{AceSrc@outlook.com} % Your email address
}
\date{\today} % Date, can be changed to a custom date

\begin{document}

\begin{frame}
\titlepage % Print the title page as the first slide
\end{frame}

\begin{frame}
\frametitle{Overview} % Table of contents slide, comment this block out to remove it
\tableofcontents % Throughout your presentation, if you choose to use \section{} and \subsection{} commands, these will automatically be printed on this slide as an overview of your presentation
\end{frame}

%----------------------------------------------------------------------------------------
%	PRESENTATION SLIDES
%----------------------------------------------------------------------------------------

%------------------------------------------------
\section{T1 predict} % Sections can be created in order to organize your presentation into discrete blocks, all sections and subsections are automatically printed in the table of contents as an overview of the talk
%------------------------------------------------

% \subsection{题目简析} % A subsection can be created just before a set of slides with a common theme to further break down your presentation into chunks

\begin{frame}
\frametitle{题目简析}

\begin{itemize}
	\item 用 $(t, x)$ 来表示命题 "$t$ 时刻第 $i$ 个人是生存状态".
	\item 用 $(t, \bar{x})$ 来表示命题 "$t$ 时刻第 $i$ 个人是死亡状态".
\end{itemize}

题目预测相当于给定若干蕴含式(implication) $A \to B$ (当且仅当 $A$ 为真命题且 $B$ 为假命题时 $A\to B$ 为假命题). 

\begin{itemize}
	\item 难兄难弟 $0,\ t,\ x,\ y$: $(t, \bar{x}) \to (t + 1, \bar{y})$, $(t + 1, y) \to (t, \bar{x})$
	\item 死神来了 $1,\ t,\ x,\ y$: $(t, x) \to (t, \bar{y})$, $(t, y) \to (t, \bar{x})$
\end{itemize}

同时题目本身自带了一些蕴含式.

\begin{itemize}
	\item $(t, \bar{x}) \to (t + 1, \bar{x})$
	\item $(t+1, x) \to (t, x)$
\end{itemize}

将 $(t, x)$ 看成点, 所有蕴含式 $A\to B$ 看成 $A$ 到 $B$ 的有向边, 我们得到了 $2SAT$ 问题中常见的有对称性的有向图. 

(对称性指若 $(t_1, x) \to^* (t_2, y)$, 就一定有 $(t_2, \bar{y}) \to^* (t_1, \bar{x})$)

\end{frame}

%------------------------------------------------

\begin{frame}
\frametitle{充要条件}
同时这个图还有一些性质, 首先这是个 DAG. 

其次只有 $(*, x) \to^{*} (*, \bar{y})$, 不会有 $(*, \bar{x}) \to^{*} (*, y)$

$Live(x, y) = 1$ 当且仅当满足以下条件

\begin{itemize}
	\item $(T+1, x) \not\to^{*} (T+1, \bar{x})$.
	\item $(T+1, y) \not\to^{*} (T+1, \bar{y})$.
	\item $(T+1, x) \not\to^{*} (T+1, \bar{y})$.
\end{itemize}

\end{frame}

%------------------------------------------------

\begin{frame}
\frametitle{直观做法}
\begin{block}{直接 BFS - 30}
就按上面说的条件, 对于第 $i$ 个人, 从 $(T + 1, i)$ 出发, 看能到达哪些 $(T+1, \bar{y})$
\end{block}

\begin{block}{离散化后 BFS - 40/50}
只考虑那些被预言提到的点. 

该做法, 由于这里机房比较快, 但虚拟机又比较慢, 加上各位的码力不同综合一下有一定分数的误差. 
\end{block}

\begin{block}{简易 bitset - 60}
每个预言会产生 $4$ 个关键点, 一共 $4m$ 个关键点, 对这 $4m$ 个点建 DAG, 用每个关键点分配一个 $n$ 位的 bitset, 
记录第 $y$ 位记录能到达 $(T+1, \bar{y})$. 

注意内存使用 $4mn/8$ bytes, 非常吃紧. 
\end{block}
\end{frame}

\begin{frame}
\frametitle{不那么的直观做法}
\begin{block}{稍微优化一下 bitset - 80}

如果一个点 $(t, x)$ 只能到达 $(t - 1, x)$, 那 $(t, x)$ 和 $(t-1, x)$ 没有任何区别, 可以合并. 

如果一个点 $(t, \bar{x})$ 只能到达 $(t+1, \bar{x})$, $(t, \bar{x})$ 和 $(t+1, \bar{x})$ 也没任何区别, 可以合并. 

然后 $(T+1, \bar{x})$ 这些点也不需要分配 bitset .

把点合并完之后, 去掉 $(T+1, \bar{x})$ 所在的点集, 在一共会剩下 $2m+n$ 个点集, 只给这些点集分配 bitset 即可.  

注意内存使用 $(2m+n)n/8$ bytes, 稍微不那么吃紧. 
\end{block}


\end{frame}

\begin{frame}
\frametitle{题解做法}

\begin{block}{使劲优化 bitset - 100}
在上面做法的基础上, 我们考虑拓扑排序, 每次选择出度为 $0$ 的点进行拓扑.

维护一个 bitset 池.  

当有若干出度为 $0$ 的点, 优先选择形如 $(t, \bar{x})$ 的点, 
如果有一个点 $(t_1, *)$ 指向 $(t, \bar{x})$, 且还未拥有一个 bitset, 我们从 bitset 池里掏出一个 bitset 给它用.

当处理完 $(t, \bar{x})$ 之后, 如果 $(t, x)$ 还未拥有 bitset, 那么我们把 $(t, \bar{x})$ 拥有的 bitset 给 $(t, x)$ 用, 
否则把 $(t, \bar{x})$ 拥有的 bitset 丢回 bitset 池. 

可以证明这样子时刻只有 $m$ 个关键点拥有 $bitset$. 

因为如果 $(t, \bar{x})$ 是一个关键点, 那么一定对应一条形如 $0,t,x,*$ 的预言, 
如果一个点 $(t, y)$ 的 bitset 是拓扑 $(t, \bar{x})$ 时得到的, 那么一定对应一条形如 $1, t, y, x$ 的预言. 

仔细思考之后发现对 $(t, x)$ 这样的点我们可以按 $t$ 从小到大选择, 这样需要 $n$ 个 bitset 作为辅助. 

内存使用 $(m+n)n/8$ bytes, 低于内存限制了. 
\end{block}

\end{frame}

%------------------------------------------------

\section{T2 neural}
\begin{frame}
\frametitle{题目简析}

大体思路是, 首先得计算出, 第 $i$ 棵树剖分成 $j$ 条链的代价 $f(i, j)$. 

在我们确定每棵树的链划分的情况下, 计算哈密顿回路个数. 

假设现在有三棵树 $A, B, C$, 第一棵树分成 $3$ 条链 $A_1, A_2, A_3$, 第二棵树分成两条链 $B_1, B_2$, 第三棵树分成两条链 $C_1, C_2$. 

在这种情况下, 我们需要计算有多少 $A_1, A_2, A_3, B_1, B_2, C_1, C_1$ 的合法排列. 

合法指第一个元素为 $A_1$, 且相邻两个元素不能来自同一棵树, 且最后一个元素不能来自树 $A$. 

比如 $A_1, B_1, A_2, C_1, A_3, C_2, B_2$ 是合法的. 

比如 $A_1, A_2, B_1, C_1, A_3, C_2, B_2$ 是不合法的. 

如果排列中第 $i$ 个元素和第 $i-1$ 个元素来自同一棵树, 那么称第 $i$ 个元素是非法元素, 比如上面的 $A_2$. 

对非法元素的个数进行容斥. 

\end{frame}

\begin{frame}
\frametitle{Part 1}

给定一棵树 $T$, 要计算将这棵树分成 $j$ 条链的方案数, 是一个典型的树形 DP(树上背包), 可以在 $O(n^2)$ 内计算出来. 

注意, 每条链是有方向的, $u \to v$ 和 $v \to u$ 是不同的方案, 因为从别的树跳到 $u$ 再走到 $v$ 和从别的树跳到 $v$ 再走到 $u$ 属于不同的哈密顿回路. 
所以做背包的时候每条点数大于 $1$ 的链答案要乘以 $2$. 

还有一个点也算一条链, 不用乘 $2$. 
\end{frame}

\begin{frame}
\frametitle{Part 2}
假设没有"第一个元素为第一棵树的第一条链, 最后一个元素不能来自第一棵树" 的限制. (等价于求解哈密顿路径个数.)

容斥部分, 最简单的想法是, 记 $g(i, j, k)$ 为考虑前 $i$ 棵树, 总共分出了 $j$ 个自由组, 非法元素的个数至少是 $k$ 的方案数. 

如何表示至少有 $k$ 个非法元素? 用类似错排问题的做法, 进行捆绑. 

比如我们现在枚举第 $x$ 棵树要分成 $y$ 条链, 我们想让这 $y$ 条链生成至少 $z$ 个非法元素, 那么我们可以把 $y$ 个数分成 $y-z$ 类自由组, 
每一类捆绑起来, 表示在排列中该类总是排在一起的. 

这和第一类斯特林数有点类似, 但和第一类斯特林数的区别在于该问题是将 $y$ 个元素分成 $y - z$ 个排列, 而不是圆排列. 

令 $S(i, j)$ 是将 $i$ 个元素分成 $j$ 个排列的方案数, 递归式是:

$S(i, j) = S(i - 1, j - 1)+(i - 1 + j)S(i - 1, j)$
\end{frame}

\begin{frame}
\frametitle{Part 2}

$g$ 最简单的方程是

$g(i, j + y - z, k + z) += g(i - 1, j, k) * f(i, y) * S(y, y - z)$

最后我们结果是 $\sum_{j} j! \times (\sum_{2 | k}f(m, j, k) - \sum_{2 | k + 1}f(m, j, k)))$.

可以看出我们没必要记录 $k$, 我们只需要记录 $k$ 的奇偶性.  

这样就得到了一个 $O(n^3)$ 的做法. 
\end{frame}

\begin{frame}
\frametitle{Part 2}


修改之后再观察下方程:
$g(i, j + y - z, (k + z)\bmod 2) += g(i - 1, j, k\bmod 2) * f(i, y) * S(y, y - z)$

注意我们并不需要枚举 $y$ 和 $z$, 我们只需要枚举 $y - z$. 

对于每个 $y - z$ 以及 $z\bmod 2$, 先对 $f(i, y) * S(y, y - z)$ 求和得到 $C(i, y - z, z\bmod 2)$. 

上面式子就变成
$g(i, j + (y - z), (k\bmod 2 + z\bmod 2)\bmod 2) += g(i - 1, j, k\bmod 2) * C(i, y - z, z\bmod 2)$. 

这样就是 $O(n^2)$ 的了. 
\end{frame}

\begin{frame}
\frametitle{Part 2}
最后我们把 "第一个元素为第一棵树的第一条链, 最后一个元素不能来自第一棵树" 限制加上, 
改写成 "最后一个自由组为第 $m$ 棵树的第一条链所在的组, 第一个自由组不能来自第 $m$ 棵树", 由于回路的性质, 这是一一对应的. 

那么只要在 DP 最后一棵树的时候稍作改变就可以了. 
\end{frame}

%------------------------------------------------
\section{T3 celebrate}
%------------------------------------------------

\begin{frame}
\frametitle{题目简析}
\begin{block}{AceSrc 曾这样说}
这大概是省选历史上最简单的 T3 了.  
\end{block}

\begin{block}{** 大学的 *sb 曾这样说}
我已经预见到知乎 "如何评价 JSOI2019 Round 2 T3 被花式拿分" 的回答了. 
\end{block}

题意很清晰, 对一个串的每个前缀求最小循环表示的开始位置, 如果有多个, 输出最小的. 

暴力分拿满是 $30$, 即对每个前缀线性求最小表示, 或者后缀结构也可以. 

各种花样后缀结构或者贪心视常数大概能有 $50/60$. 

后缀数组的 $O(n\log n)$ 算法视常数可能会 T 最后几个点. 

标程复杂度 $O(n\log n)$, 算法 z-algorithm, 就是俗称的扩展 KMP.
\end{frame}

\begin{frame}
\frametitle{性质 1}

假设我们现在考虑前缀 $S[1..k]$, 我们考虑有哪些位置可能称为 $S[1..k]$ 最小循环表示的开始位置, 我们称这些位置为候选点, 
我们需要使用一些性质减少候选点的个数.  

设 $n$ 是字符串长度. 

\begin{block}{性质 1}
假设两个位置 $i < j$. 如果 $common-prefix(S[i..n], S[j..n]) \le k - j$, 那么 $i, j$ 之间肯定有一个不是候选点. 
\end{block}

假设你已经利用这个性质, 得到 $S[1..k - 1]$ 的候选点集 $P$, 对于两个数 $i \in P, j \in P, i < j$, 肯定有
$common-prefix(S[i..n], S[j..n]) > (k - 1) - j$, 我们现在想知道是否有 $common-prefix(S[i..n], S[j..n]) > k - j$, 
那么我们只需要比较 $S[i + k - j]$ 和 $S[k]$ 即可. 

用这个性质大概能有 $50/60$ 分. 

\end{frame}

\begin{frame}
\frametitle{性质 2}
还是假设我们现在考虑前缀 $S[1..k]$. 

\begin{block}{性质 1}
对于两个点 $i < j$, 假设 $common-prefix(S[i..n], S[j..n]]) > k - j$, 
如果有 $k - j \ge j - i$, 那么 $j$ 不是候选点. 
\end{block}

这个性质是有最小循环表示的某个性质得来的, 假设串 $S=S_1S_1S_2$, 其中 $S_1, S_2$ 是子串. 

要么有 $S_1 S_1 S_2 < S_1 S_2 S1 < S_2 S_1 S_1$, 

要么有 $S_1 S_1 S_2 > S_1 S_2 S1 > S_2 S_1 S_1$

如果有 $k - j \ge j - i$, 那么 $S[1..k]$ 就肯定形如 $ABBC$, 其中 $A, B, C$ 是子串. 

这个性质直接保证了, 候选点个数不会超过 $\log n$ 个. 
\end{frame}

\begin{frame}
\frametitle{解决!}
对于候选点, 我们直接使用 z-algorithm 就能比较循环同构串的字典序大小. 

如果用后缀数组做这一步, 有可能超时. 
\end{frame}
%------------------------------------------------

\begin{frame}
\Huge{\centerline{The End}}
\end{frame}

%----------------------------------------------------------------------------------------

\end{document}
